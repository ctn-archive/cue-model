% uWaterloo Thesis Template for LaTeX 
% Last Updated Nov 4, 2016 by Stephen Carr, IST Client Services
% FOR ASSISTANCE, please send mail to rt-IST-CSmathsci@ist.uwaterloo.ca

% Effective October 2006, the University of Waterloo 
% requires electronic thesis submission. See the uWaterloo thesis regulations at
% https://uwaterloo.ca/graduate-studies/thesis.

% DON'T FORGET TO ADD YOUR OWN NAME AND TITLE in the "hyperref" package
% configuration below. THIS INFORMATION GETS EMBEDDED IN THE PDF FINAL PDF DOCUMENT.
% You can view the information if you view Properties of the PDF document.

% Many faculties/departments also require one or more printed
% copies. This template attempts to satisfy both types of output. 
% It is based on the standard "book" document class which provides all necessary 
% sectioning structures and allows multi-part theses.

% DISCLAIMER
% To the best of our knowledge, this template satisfies the current uWaterloo requirements.
% However, it is your responsibility to assure that you have met all 
% requirements of the University and your particular department.
% Many thanks for the feedback from many graduates that assisted the development of this template.

% -----------------------------------------------------------------------

% By default, output is produced that is geared toward generating a PDF 
% version optimized for viewing on an electronic display, including 
% hyperlinks within the PDF.
 
% E.g. to process a thesis called "mythesis.tex" based on this template, run:

% pdflatex mythesis	-- first pass of the pdflatex processor
% bibtex mythesis	-- generates bibliography from .bib data file(s)
% makeindex         -- should be run only if an index is used 
% pdflatex mythesis	-- fixes numbering in cross-references, bibliographic references, glossaries, index, etc.
% pdflatex mythesis	-- fixes numbering in cross-references, bibliographic references, glossaries, index, etc.

% If you use the recommended LaTeX editor, Texmaker, you would open the mythesis.tex
% file, then click the PDFLaTeX button. Then run BibTeX (under the Tools menu).
% Then click the PDFLaTeX button two more times. If you have an index as well,
% you'll need to run MakeIndex from the Tools menu as well, before running pdflatex
% the last two times.

% N.B. The "pdftex" program allows graphics in the following formats to be
% included with the "\includegraphics" command: PNG, PDF, JPEG, TIFF
% Tip 1: Generate your figures and photos in the size you want them to appear
% in your thesis, rather than scaling them with \includegraphics options.
% Tip 2: Any drawings you do should be in scalable vector graphic formats:
% SVG, PNG, WMF, EPS and then converted to PNG or PDF, so they are scalable in
% the final PDF as well.
% Tip 3: Photographs should be cropped and compressed so as not to be too large.

% To create a PDF output that is optimized for double-sided printing: 
%
% 1) comment-out the \documentclass statement in the preamble below, and
% un-comment the second \documentclass line.
%
% 2) change the value assigned below to the boolean variable
% "PrintVersion" from "false" to "true".

% --------------------- Start of Document Preamble -----------------------

% Specify the document class, default style attributes, and page dimensions
% For hyperlinked PDF, suitable for viewing on a computer, use this:
\documentclass[letterpaper,12pt,titlepage,oneside,final]{scrbook}
 
% For PDF, suitable for double-sided printing, change the PrintVersion variable below
% to "true" and use this \documentclass line instead of the one above:
%\documentclass[letterpaper,12pt,titlepage,openright,twoside,final]{book}


% This package allows if-then-else control structures.
\usepackage{ifthen}
\newboolean{PrintVersion}
\setboolean{PrintVersion}{false} 
% CHANGE THIS VALUE TO "true" as necessary, to improve printed results for hard copies
% by overriding some options of the hyperref package below.

\usepackage{fontspec}
\usepackage{microtype}
\usepackage{amsmath,amssymb,amstext}
\usepackage{bm}
\usepackage{commath}
\usepackage[backend=biber]{biblatex}
\usepackage[]{graphicx}
\usepackage{interval}
\usepackage{siunitx}
\usepackage{tikz}

\usetikzlibrary{graphs}
\usetikzlibrary{nef}
\usetikzlibrary{quotes}

\addbibresource{uw-ethesis.bib}

\graphicspath{{../../figures/}}
\intervalconfig{separator symbol={,\,}}

% Hyperlinks make it very easy to navigate an electronic document.
% In addition, this is where you should specify the thesis title
% and author as they appear in the properties of the PDF document.
% Use the "hyperref" package 
% N.B. HYPERREF MUST BE THE LAST PACKAGE LOADED; ADD ADDITIONAL PKGS ABOVE
\usepackage[pagebackref=false]{hyperref} % with basic options
		% N.B. pagebackref=true provides links back from the References to the body text. This can cause trouble for printing.
\hypersetup{
    plainpages=false,       % needed if Roman numbers in frontpages
    unicode=false,          % non-Latin characters in Acrobat’s bookmarks
    pdftoolbar=true,        % show Acrobat’s toolbar?
    pdfmenubar=true,        % show Acrobat’s menu?
    pdffitwindow=false,     % window fit to page when opened
    pdfstartview={FitH},    % fits the width of the page to the window
    pdftitle={uWaterloo\ LaTeX\ Thesis\ Template},    % title: CHANGE THIS TEXT!
    pdfauthor={Jan Gosmann},    % author: CHANGE THIS TEXT! and uncomment this line
%    pdfsubject={Subject},  % subject: CHANGE THIS TEXT! and uncomment this line
%    pdfkeywords={keyword1} {key2} {key3}, % list of keywords, and uncomment this line if desired
    pdfnewwindow=true,      % links in new window
    colorlinks=true,        % false: boxed links; true: colored links
    linkcolor=blue,         % color of internal links
    citecolor=green,        % color of links to bibliography
    filecolor=magenta,      % color of file links
    urlcolor=cyan           % color of external links
}
\ifthenelse{\boolean{PrintVersion}}{   % for improved print quality, change some hyperref options
\hypersetup{	% override some previously defined hyperref options
%    colorlinks,%
    citecolor=black,%
    filecolor=black,%
    linkcolor=black,%
    urlcolor=black}
}{} % end of ifthenelse (no else)

\usepackage[automake,toc,abbreviations]{glossaries-extra} % Exception to the rule of hyperref being the last add-on package

\usepackage[capitalise]{cleveref}

% Setting up the page margins...
% uWaterloo thesis requirements specify a minimum of 1 inch (72pt) margin at the
% top, bottom, and outside page edges and a 1.125 in. (81pt) gutter
% margin (on binding side). While this is not an issue for electronic
% viewing, a PDF may be printed, and so we have the same page layout for
% both printed and electronic versions, we leave the gutter margin in.
% Set margins to minimum permitted by uWaterloo thesis regulations:
\setlength{\marginparwidth}{0pt} % width of margin notes
% N.B. If margin notes are used, you must adjust \textwidth, \marginparwidth
% and \marginparsep so that the space left between the margin notes and page
% edge is less than 15 mm (0.6 in.)
\setlength{\marginparsep}{0pt} % width of space between body text and margin notes
\setlength{\evensidemargin}{0.125in} % Adds 1/8 in. to binding side of all 
% even-numbered pages when the "twoside" printing option is selected
\setlength{\oddsidemargin}{0.125in} % Adds 1/8 in. to the left of all pages
% when "oneside" printing is selected, and to the left of all odd-numbered
% pages when "twoside" printing is selected
\setlength{\textwidth}{6.375in} % assuming US letter paper (8.5 in. x 11 in.) and 
% side margins as above
\raggedbottom

% The following statement specifies the amount of space between
% paragraphs. Other reasonable specifications are \bigskipamount and \smallskipamount.
%\setlength{\parskip}{\medskipamount}

% The following statement controls the line spacing.  The default
% spacing corresponds to good typographic conventions and only slight
% changes (e.g., perhaps "1.2"), if any, should be made.
%\renewcommand{\baselinestretch}{1} % this is the default line space setting

% By default, each chapter will start on a recto (right-hand side)
% page.  We also force each section of the front pages to start on 
% a recto page by inserting \cleardoublepage commands.
% In many cases, this will require that the verso page be
% blank and, while it should be counted, a page number should not be
% printed.  The following statements ensure a page number is not
% printed on an otherwise blank verso page.
\let\origdoublepage\cleardoublepage
\newcommand{\clearemptydoublepage}{%
  \clearpage{\pagestyle{empty}\origdoublepage}}
\let\cleardoublepage\clearemptydoublepage

\newcommand{\mat}[1]{#1}
\newcommand{\vc}[1]{\bm{#1}}
\newcommand{\ped}[1]{{\mathrm{#1}}}
\newcommand{\Tr}{^{\top}}

\newcommand{\pop}[1]{\textit{#1}}

% Define Glossary terms (This is properly done here, in the preamble. Could be \input{} from a file...)
% Main glossary entries -- definitions of relevant terminology
\newglossaryentry{computer}
{
name=computer,
description={A programmable machine that receives input data,
               stores and manipulates the data, and provides
               formatted output}
}

% Nomenclature glossary entries -- New definitions, or unusual terminology
\newglossary*{nomenclature}{Nomenclature}
\newglossary*{symbols}{List of Symbols}
\makeglossaries
%\newglossaryentry{dingledorf}
%{
%type=nomenclature,
%name=dingledorf,
%description={A person of supposed average intelligence who makes incredibly brainless misjudgments}
%}

% List of Abbreviations (abbreviations type is built in to the glossaries-extra package)
\newabbreviation{aaaaz}{AAAAZ}{American Association of Amature Astronomers and Zoologists}

% List of Symbols
\newcommand{\addsym}[4]{\newglossaryentry{#2}{sort={#2},type=symbols,name={\ensuremath{#3}},description={#4}}\glsadd{#2}\newcommand{#1}{\ensuremath{#3}}}
\addsym{\ctx}{c}{\vc{c}}{TCM context vector}
\addsym{\ctxin}{cin}{\vc{c}^\ped{IN}}{TCM input context vector}
\addsym{\ltwo}{ltwo}{l^2}{Euclidean norm} 
\addsym{\tcmbeta}{beta}{\beta}{TCM beta parameter}
\addsym{\Heavi}{heavi}{\Theta}{Heaviside function}
\addsym{\mft}{mft}{\mat{M}_\ped{FT}}{TCM item-to-context association matrix}
\addsym{\mtf}{mtf}{\mat{M}_\ped{TF}}{TCM context-to-item association matrix}


%======================================================================
%   L O G I C A L    D O C U M E N T -- the content of your thesis
%======================================================================
\begin{document}

% For a large document, it is a good idea to divide your thesis
% into several files, each one containing one chapter.
% To illustrate this idea, the "front pages" (i.e., title page,
% declaration, borrowers' page, abstract, acknowledgements,
% dedication, table of contents, list of tables, list of figures,
% nomenclature) are contained within the file "uw-ethesis-frontpgs.tex" which is
% included into the document by the following statement.
%----------------------------------------------------------------------
% FRONT MATERIAL
%----------------------------------------------------------------------
% T I T L E   P A G E
% -------------------
% Last updated Nov 1, 2016, by Stephen Carr, IST-Client Services
% The title page is counted as page `i' but we need to suppress the
% page number.  We also don't want any headers or footers.
\pagestyle{empty}
\pagenumbering{roman}

% The contents of the title page are specified in the "titlepage"
% environment.
\begin{titlepage}
        \begin{center}
        \vspace*{1.0cm}

        \Huge
        {\textbf{An Integrated Model of Context, Short-Term, and Long-Term Memory}}

        \vspace*{1.0cm}

        \normalsize
        by \\

        \vspace*{1.0cm}

        \Large
        Jan Gosmann \\

        \vspace*{3.0cm}

        \normalsize
        A thesis \\
        presented to the University of Waterloo \\ 
        in fulfillment of the \\
        thesis requirement for the degree of \\
        Doctor of Philosophy \\
        in \\
        Systems Design Engineering \\

        \vspace*{2.0cm}

        Waterloo, Ontario, Canada, 2017 \\

        \vspace*{1.0cm}

        \copyright\ Jan Gosmann 2017 \\
        \end{center}
\end{titlepage}

% The rest of the front pages should contain no headers and be numbered using Roman numerals starting with `ii'
\pagestyle{plain}
\setcounter{page}{2}

\cleardoublepage % Ends the current page and causes all figures and tables that have so far appeared in the input to be printed.
% In a two-sided printing style, it also makes the next page a right-hand (odd-numbered) page, producing a blank page if necessary.
 


% D E C L A R A T I O N   P A G E
% -------------------------------
  % The following is a sample Delaration Page as provided by the GSO
  % December 13th, 2006.  It is designed for an electronic thesis.
  \noindent
I hereby declare that I am the sole author of this thesis. This is a true copy of the thesis, including any required final revisions, as accepted by my examiners.

  \bigskip
  
  \noindent
I understand that my thesis may be made electronically available to the public.

\cleardoublepage

% A B S T R A C T
% ---------------

\begin{center}\textbf{Abstract}\end{center}

This is the abstract.

Vulputate minim vel consequat praesent at vel iusto et, ex delenit, esse euismod luptatum augue ut sit et eu vel augue autem feugiat, quis ad dolore. Nulla vel, laoreet lobortis te commodo elit qui aliquam enim ex iriure ea ullamcorper nostrud lorem, lorem laoreet eu ex ut vel in zzril wisi quis. Nisl in autem praesent dignissim, sit vel aliquam at te, vero dolor molestie consequat.

Tation iriure sed wisi feugait odio dolore illum duis in accumsan velit illum consequat consequat ipsum molestie duis duis ut ullamcorper. Duis exerci odio blandit vero dolore eros odio amet et nisl in nostrud consequat iusto eum suscipit autem vero. Iusto dolore exerci, ut erat ex, magna in facilisis duis amet feugait augue accumsan zzril delenit aliquip dignissim at. Nisl molestie nibh, vulputate feugait nibh luptatum ea delenit nostrud dolore minim veniam odio volutpat delenit nulla accumsan eum vero ullamcorper eum. Augue velit veniam, dolor, exerci ea feugiat nulla molestie, veniam nonummy nulla dolore tincidunt, consectetuer dolore nulla ipsum commodo.

At nostrud lorem, lorem laoreet eu ex ut vel in zzril wisi. Suscipit consequat in autem praesent dignissim, sit vel aliquam at te, vero dolor molestie consequat eros tation facilisi diam dolor. Odio luptatum dolor in facilisis et facilisi et adipiscing suscipit eu iusto praesent enim, euismod consectetuer feugait duis. Odio veniam et iriure ad qui nonummy aliquip at qui augue quis vel diam, nulla. Autem exerci tation iusto, hendrerit et, tation esse consequat ut velit te dignissim eu esse eros facilisis lobortis, lobortis hendrerit esse dignissim nisl. Nibh nulla minim vel consequat praesent at vel iusto et, ex delenit, esse euismod luptatum.

Ut eum vero ullamcorper eum ad velit veniam, dolor, exerci ea feugiat nulla molestie, veniam nonummy nulla. Elit tincidunt, consectetuer dolore nulla ipsum commodo, ut, at qui blandit suscipit accumsan feugiat vel praesent. In dolor, ea elit suscipit nisl blandit hendrerit zzril. Sit enim, et dolore blandit illum enim duis feugiat velit consequat iriure sed wisi feugait odio dolore illum duis. Et accumsan velit illum consequat consequat ipsum molestie duis duis ut ullamcorper nulla exerci odio blandit vero dolore eros odio amet et.

In augue quis vel diam, nulla dolore exerci tation iusto, hendrerit et, tation esse consequat ut velit. Duis dignissim eu esse eros facilisis lobortis, lobortis hendrerit esse dignissim nisl illum nulla minim vel consequat praesent at vel iusto et, ex delenit, esse euismod. Nulla augue ut sit et eu vel augue autem feugiat, quis ad dolore te vel, laoreet lobortis te commodo elit qui aliquam enim ex iriure. Ut ullamcorper nostrud lorem, lorem laoreet eu ex ut vel in zzril wisi quis consequat in autem praesent dignissim, sit vel. Dolore at te, vero dolor molestie consequat eros tation facilisi diam. Feugait augue luptatum dolor in facilisis et facilisi et adipiscing suscipit eu iusto praesent enim, euismod consectetuer feugait duis vulputate veniam et.

Ad eros odio amet et nisl in nostrud consequat iusto eum suscipit autem vero enim dolore exerci, ut. Esse ex, magna in facilisis duis amet feugait augue accumsan zzril. Lobortis aliquip dignissim at, in molestie nibh, vulputate feugait nibh luptatum ea delenit nostrud dolore minim veniam odio. Euismod delenit nulla accumsan eum vero ullamcorper eum ad velit veniam. Quis, exerci ea feugiat nulla molestie, veniam nonummy nulla. Elit tincidunt, consectetuer dolore nulla ipsum commodo, ut, at qui blandit suscipit accumsan feugiat vel praesent.

Dolor zzril wisi quis consequat in autem praesent dignissim, sit vel aliquam at te, vero. Duis molestie consequat eros tation facilisi diam dolor augue. Dolore dolor in facilisis et facilisi et adipiscing suscipit eu iusto praesent enim, euismod consectetuer feugait duis vulputate.

\cleardoublepage

% A C K N O W L E D G E M E N T S
% -------------------------------

\begin{center}\textbf{Acknowledgements}\end{center}

I would like to thank all the little people who made this thesis possible.
\cleardoublepage

% D E D I C A T I O N
% -------------------

\begin{center}\textbf{Dedication}\end{center}

This is dedicated to the one I love.
\cleardoublepage

% T A B L E   O F   C O N T E N T S
% ---------------------------------
\renewcommand\contentsname{Table of Contents}
\tableofcontents
\cleardoublepage
%\phantomsection    % allows hyperref to link to the correct page

% L I S T   O F   T A B L E S
% ---------------------------
\addchap{List of Tables}
%\addcontentsline{toc}{chapter}{List of Tables}
\listoftables
\cleardoublepage
%\phantomsection		% allows hyperref to link to the correct page

% L I S T   O F   F I G U R E S
% -----------------------------
\addchap{Lisf of Figures}
%\addcontentsline{toc}{chapter}{List of Figures}
\listoffigures
\cleardoublepage
%\phantomsection		% allows hyperref to link to the correct page

% GLOSSARIES (Lists of definitions, abbreviations, symbols, etc. provided by the glossaries-extra package)
% -----------------------------
\printglossaries
\cleardoublepage
%\phantomsection		% allows hyperref to link to the correct page

% Change page numbering back to Arabic numerals
\pagenumbering{arabic}

 

%----------------------------------------------------------------------
% MAIN BODY
%----------------------------------------------------------------------
% Because this is a short document, and to reduce the number of files
% needed for this template, the chapters are not separate
% documents as suggested above, but you get the idea. If they were
% separate documents, they would each start with the \chapter command, i.e, 
% do not contain \documentclass or \begin{document} and \end{document} commands.
%======================================================================
\chapter{Context update}

The context update network has to approximate Equation~TODO which, as a reminder, is restated here:
\begin{equation}
    \ctx_i = \rho_i \ctx_{i-1} + \tcmbeta \ctxin_i\,\text{.} \label{eqn:ctx-update}
\end{equation}
Different methods of approximating this equation can be thought of and in the following I will describe four methods of which only one was successful in matching the data.
Even though most of these methods have been unsuccessful it is instructive to see why these methods failed to match the data as this demonstrates which features of the mathematical TCM formulation are relevant and which are non-relevant side-effects of a particular formulation.

\section{Boundend integrator}
Equation~\ref{eqn:ctx-update} assumes discrete steps, but for a neural implementation a continuous formulation is more natural and given by
\begin{equation}
    \od{\ctx}{t} = (\bar{\rho} - 1) \ctx + \bar{\tcmbeta} \ctxin\,\text{.}
\end{equation}
This equation is easily implemented with a neural integrator for a constant $\bar{rho}$ and $\bar{\tcmbeta}$.
However, there is no limit on the integration of $\ctxin$ anymore.
To add at most $\tcmbeta \ctxin$ to the context $\ctx$ we can gate the input to the integrator and add a network computing the dot product between $\ctx$ and $\ctxin$.
After thresholding it at $\tcmbeta$ it can be used to suppress the input by inhibiting the gate (TODO figure).
Furthermore, $\bar{\rho}$ needs to be adjusted to keep the unit length of $\ctx$.
To do so, we can project $c$ to another population \pop{downscale} which projects back to the integrator with a transform of $\gamma = -0.1$.
Picking a $\gamma$ closer to zero will allow the $\vc c$ vector exceed unit length by a larger amount while the integrator receives input and will increase the time required to settle back to unit length, whereas a large magnitude of $\gamma$ can lead to oscillatory behaviour.
The \pop{downscale} population needs to be controlled to only provide the inhibitory input to the integrator as long as $\norm{\ctx} > 1$.
This is achieved by decoding the the length of $\vc c$ from the integrator and thresholding it at $1$.
As long as the threshold is not exceeded \pop{downscale} will be inhibited.
\begin{figure}
    \begin{tikzpicture}[
        ext/.style={draw=none},
        net/.style={draw, rounded corners=0.5em},
        ens/.style={draw, circle, inner sep=0.25ex},
        rect/.style={draw, diamond, inner sep=0.25ex},
        state/.style={draw, fill=white, circle, inner sep=0.05em, double copy shadow={opacity=0.6, shadow yshift=-0.3ex, shadow xshift=0.3ex}},
        recurrent/.style={loop above, min distance=2em, in=120, out=60}
        ]
        \graph[grow right sep=15mm, branch down sep=15mm, nodes={anchor=center, minimum width=1.5em, minimum height=1.5em}, edge quotes={above, node font=\small}] {
            in/\ctxin [ext] -!- {
                gate/ [state] -> ["$\bar{\tcmbeta}$"] integrator/\ctx [state] -> out/ [ext],
                threshold/ [rect],
                dot [net]
            },
            in -> gate,
            in -> dot -> threshold -> ["$\Heavi(x - \bar{\tcmbeta})$" {rotate=90}] gate,
            integrator -> dot,
            integrator -> [recurrent, "$\bar{\rho}$" above] integrator
        };
    \end{tikzpicture}
    \caption{TODO}
\end{figure}

This network fulfills two criteria of the context update equation: the new context is added in with a strength of at most $\tcmbeta$ and the context vector is kept at unit length.
Unfortunately, it does not give the desired match to human data.
The main reason for that is that TODO (old context not preserved by increasing length and renormalizing? Context input not orthogonal?)

\section{Alternating update of two memories}
As the continuous update of a single integrator does not yield the desired results, it is necessary to approximate Equation~\ref{eqn:ctx-update} more closely.
This can be done with memory populations (TODO describe in previous chapter) that allow to update the stored value quickly instead of continuously shifting the stored vector to the target vector.
One memory population is used to store the old context $\ctx_{i-1}$.
This old context and the input context $\ctxin_i$ can be used to update the other memory population with appropriate weightings $\rho = \sqrt{1 - \tcmbeta^2}$ and $\tcmbeta$.
Note that the $\rho$ weighting is an approximation using the fact that the old context and input will be almost orthogonal in a high-dimensional space.

Once the second memory population has been updated, the input gate can be closed and the first memory population can be updated, thus making the updated context the ``old'' context.
However, the neural control of this switching between updating the two memory populations is not trivial.
In a first approximation one can use the dot product between the input and the current context to switch the updating once a threshold of $\tcmbeta$ is exceeded.

Trying to determine if dot product is changing, for what?
Is there a problem with the dot product changing too early once the old context gets updated?

orthogonality violated on recall

\section{Three memory context update}
To solve the problems with updating two memories alternatingly a third memory can be introduced.
That way the current context can be updated as before with a combination of $\rho \ctx_{i-1} + \tcmbeta \ctxin_i$.
Once the update is done, the third intermediary buffer gets updated before the old context gets updated from this buffer (and updates to the current context are also allowed again). TODO, why exactly solves this problems?

In this scheme it is easy to determine that the current context has been updated by using a dot product between the current context and the input to the current context.
If it exceeds one, the update is finished and the intermediary buffer can be updated.
TODO what problem solves this compared to the dot product of $\ctxin_i$ and the current context thresholded at $\tcmbeta$?

Explain remaining problem of growing (?) context.

\section{Externally controlled three memory context update with downscale}
All approaches to determine required context updates based on vector similarity will fail because the similarity of $\ctxin_i$ and $\ctx_{i-1}$ is not known beforehand and can vary widely depending on what contexts are recalled.
Thus, for a properly working context update in the TCM model, the update process has to be controlled by an external control signal (TODO reference other chapter).
The three memory structure works very well in that case with one exception, that with similar vectors the context vector can exceed unit length.
This can be fixed by introducing a downscaling network as in TODO\@.
This network fulfills all required criteria for the TCM context update.

It leads to a number of predictions.
First, the update of the context signal is not directly regulated by the input, but externally controlled. Second, there are neural populations that will start representing the current context in succession. Third, there are neural populations that become active only when highly similar contexts are retrieved (as only in that case the context vector exceeds unit length and will activate the downscale population).

\chapter{Association Matrix Learning}

The TCM requires two association matrices, $\mat M_\ped{TF}$ and $\mat M_\ped{FT}$, to be updated.
To translate this into neurons an appropriate learning rule, the association matrix learning rule (AML), has to be derived.
The TCM gives the update of such an association matrix as
\begin{eqnarray}
    \mat M_{i+1} =& \mat M_i + \Delta \mat M_i \\
    \Delta \mat M_i =& \vc v_i \vc u_i\Tr
\end{eqnarray}
for adding an association from $\vc a$ to $\vc b$.
The association matrix after $n$ updates can be expressed as
\begin{equation}
    \mat M_n = M_0 + \sum_{i=1}^{n} \Delta \mat M_i = M_0 + \sum_{i=1}^n \vc v_i \vc u_i\Tr \text{.}
\end{equation}
This allows us to express the neural connection weights after learning $n$ associations as
\begin{equation}
    \mat W = \mat E \mat M_n \mat D = \mat E \mat M_0 \mat D + E \sum_{i=1}^n \mat \vc v_i \vc u_i\Tr \mat D
\end{equation}
where $\mat E$ is the post-synaptic encoder matrix and $\mat D$ are the pre-synaptic decoders of the identity function.
This equation gives us some important information on how the learning of such association matrices can be implemented.
First, preexisting weights can be implemented as a transform on a normal neural connection that is kept constant.
Second, all the weight changes can be collapsed into decoder changes.
Thus, we need the AML to implement the decoder change given by
\begin{equation}
    \Delta \tilde{\mat D} = \vc v_i \vc u_i\Tr \mat D
\end{equation}
where $\tilde{\mat D}$ is the matrix of learned decoders.

To implement this within a neural network, the discrete equation has to be converted into continuous form:
\begin{equation}
    \od{\tilde{\mat D}}{t} = \eta \vc v(t) \vc u(t)\Tr \mat D
\end{equation}
with learning rate $\eta$.
This equation can be directly implemented with the NEF and thus realized with spiking neurons.
That alone, however, does not ensure the biological plausibility as any mathematical formulations of synaptic weight changes could be implemented with the NEF\@.
To get a better sense of the biological plausibility it is useful to obtain a formulation of the learning rule in terms of neural activities by replacing the decoded values, yielding
\begin{eqnarray}
    \od{\tilde{\mat D}}{t} =& \eta (D_v a_v(t)) (D a_u(t))\Tr D \\
    =& \eta D_v (a_v(t) a_u\Tr(t)) D\Tr D \text{.}
\end{eqnarray}
Here, $D\Tr D$ gives a correlation matrix of neurons representing the same dimension and $a_v(t) a_u\Tr(t)$ can be interpreted as a \emph{modulatory Hebbian term}.
Unlike a standard Hebbian term it does not connect neurons that fire together, but the combined firing of two neurons modulates the connection of one of these neurons to a third population (see Fig.~TODO).
If the population $u$ projects to is used as the $v$ input population, this rule will become truly Hebbian and neurons that fire together will connect with respect to the weighting introduced by $D_v$ and $D\Tr D$.
Note that in contrast to many other learning rules AML does not produce destructive interference due to the correlation matrix term.

Note that the AML, like many other Hebbian-style learning rules, allows weights to grow without bound.
By introducing a factor of $1 - v(t)\Tr \hat{v}(t)$ this can be prevented, but similar to other weight normalizations it introduces the need for each weight to have access to the global population activity and weights as $\hat{v}(t) = \tilde{\mat D} a_u(t)$.
Such global dependencies are ofter criticized for not being biological plausible.
As such, I decided to take a slightly different approach with an equivalent effect.
Instead of including the dot product $v(t)\Tr \hat{v}(t)$ in the learning rule, it can be computed by another neural population and the result can be used to inhibit the population providing $v$.
Once fully inhibited $a_v(t)$ will be all-zero and thus prevent further weight changes.


\chapter{Acknowledgements}
IK
%----------------------------------------------------------------------
% END MATERIAL
%----------------------------------------------------------------------

% B I B L I O G R A P H Y
% -----------------------

% The following statement selects the style to use for references.  It controls the sort order of the entries in the bibliography and also the formatting for the in-text labels.
%\bibliographystyle{plain}
% This specifies the location of the file containing the bibliographic information.  
% It assumes you're using BibTeX (if not, why not?).
\cleardoublepage % This is needed if the book class is used, to place the anchor in the correct page,
                 % because the bibliography will start on its own page.
                 % Use \clearpage instead if the document class uses the "oneside" argument
\phantomsection  % With hyperref package, enables hyperlinking from the table of contents to bibliography             
% The following statement causes the title "References" to be used for the bibliography section:
\renewcommand*{\bibname}{References}

% Add the References to the Table of Contents
\addcontentsline{toc}{chapter}{\textbf{References}}

% Tip 5: You can create multiple .bib files to organize your references.  Just 
% list them all in the \bibliogaphy command, separated by commas (no spaces).

% The following statement causes the specified references to be added to the bibliography% even if they were not 
% cited in the text. The asterisk is a wildcard that causes all entries in the bibliographic database to be included (optional).
\nocite{*}
\printbibliography

% The \appendix statement indicates the beginning of the appendices.
\appendix
% Add a title page before the appendices and a line in the Table of Contents
\chapter*{APPENDICES}
\addcontentsline{toc}{chapter}{APPENDICES}
%======================================================================
\chapter[PDF Plots From Matlab]{Matlab Code for Making a PDF Plot}
\label{AppendixA}
% Tip 4: Example of how to get a shorter chapter title for the Table of Contents 
%======================================================================

\end{document}
